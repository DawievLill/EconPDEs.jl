\documentclass[english]{article}
\usepackage[T1]{fontenc}
\usepackage[latin9]{inputenc}
\usepackage{color}
\usepackage{amsmath}
\usepackage{amssymb}
\usepackage{esint}
\usepackage{babel}
\usepackage[round]{natbib}
\usepackage{color,hyperref}
\definecolor{darkblue}{rgb}{0.0,0.0,0.3}
\hypersetup{colorlinks,breaklinks,
linkcolor=darkblue,urlcolor=darkblue,
anchorcolor=darkblue,citecolor=darkblue}
\usepackage[capitalise, noabbrev]{cleveref}

\begin{document}
	\title{Solving PDEs Associated with Economic Models}
	\author{\large{\textsc{Matthieu Gomez \thanks{I thank Valentin Haddad, Ben Moll, and Dejanir Silva for useful discussions.}}}}
	\date{\today}
	\maketitle
	This package \href{https://github.com/matthieugomez/EconPDEs.jl}{EconPDEs.jl} introduces a fast and robust way to solve systems of PDEs + algebraic equations (i.e. DAEs) associated with economic models. This note details the underlying algorithm. 
	\paragraph{Difference with \citet{achdou2014heterogeneous}}
	 \citet{achdou2014heterogeneous} focus on quasi-linear PDEs of the form
	\begin{align*}
		0&=f_1(V)  + f_2(x) \partial_x V  + f_3(x) \partial_{xx} V
	\end{align*}
	In contrast, the package solves non-linear PDEs of the form
		\begin{align*}
			0&=f_1(V)  + f_2(x, \partial_x V)  + f_3(x, \partial_x V) \partial_{xx} V
		\end{align*}

	

	\paragraph{Step 1: Write Finite Difference Scheme}
	The system of PDEs is written on a state space grid and derivatives are substituted by finite difference approximations. As in \citet{achdou2014heterogeneous},  first order derivatives are upwinded. This allows to naturally handle boudnary counditions at the frontiers of the state space. This also tends to make the scheme monotonous.

	\paragraph{Step 2: Solve Finite Difference Scheme}

	Denote $V$  the solution of the PDE and denote $F(V)$ the finite difference scheme corresponding to a model. The goal is to find $V$ such that $F(V) = 0$. The package includes a solver especially written for these finite different schemes. This method is most similar to a method used in the fluid dynamics literature. In this context, it is called the Pseudo-Transient Continuation method, and is denoted $\Psi tc$. Formal conditions for the convergence of the algorithm are given in  \citet{kelley1998convergence}.\par
	To understand the intuition for the method, note that the existing literature in economics solves for $V$ using using one of the two methods:

	\begin{enumerate}
		\item Non-linear solver. The method solves for the non-linear system $F (V ) = 0$. A Newton-Raphson update takes the form
		\begin{align}
			\label{1}
			0 &= F(y_{t}) + J_{F}(v_t) (v_{t+1} - v_t)
		\end{align}
		The issue with this method is that it requires the initial guess to be sufficiently close to the solution.\footnote{This method is usded, for instance, by \citet{garleanu2015young}}
		\item ODE solver . The method solves for the ODE $F(V) = \dot{V}$. The solution of $F(V)=0$ is obtained with $T\rightarrow +\infty$. \footnote{See, for instance, \citet{ditellabalance}, \citet{silva2015risk}.}
		With a simple explicit Euler method, updates take the form
		\begin{align}
			\label{2}
			0&= F(v_t) - \frac{1}{\Delta} (v_{t+1} -y_{t})
		\end{align}
		This method tends to be slow, and does not always converge, depending on the ODE chosen.
	\end{enumerate}
	I propose to solve for $V$ using a fully implicit Euler method.  Updates take the form 
	\begin{align*}
		\forall t \leq T \hspace{1cm} 0&= F(v_{t+1}) - \frac{1}{\Delta}(v_{t+1} -y_{t})
	\end{align*}
	Each time step now requires to solve a non-linear equation. I solve this non-linear equation using a Newton-Raphson method. These inner iterations therefore take the form
	\begin{align}
		\label{inner}
		\forall i \leq I \hspace{1cm}	0 &= F(y_{t}^i) - \frac{1}{\Delta}(y_{t}^{i} -y_{t}) + (J_{F}(v_t^i) -  \frac{1}{\Delta})(y^{i+1}_{t} - v_t^i)
	\end{align}
	We know that the Newton-Raphson method converges if the initial guess is close enough to the solution. Since $y_{t}$ converges towards $v_{t+1}$ as $\Delta$ tends to zero, one can always choose $\Delta$ low enough so that the inner steps converge.Therefore, I adjust $\Delta$ as follows. If the inner iterations do not converge, I decrease $\Delta$. When the inner iteration converges, I increase $\Delta$. \par
	The update \cref{inner} can be see as weighted average of the Newton-Raphson step \cref{1} and of the explicit Euler step \cref{2}.  After a few sucessful implicit time steps, $\Delta$ is large and therefore the algorithm becomes like Newton-Rapshon. In particular, the convergence is quadratic around the solution. \par
	The algorithm with $I=1$ and $\Delta$ constant corresponds to \citet{achdou2014heterogeneous}. Allowing $I > 1$ and adjusting $\Delta$  are important to ensure convergence in case on non-linear PDEs.




		
	\bibliography{bib}
	\bibliographystyle{aer}
\end{document}
